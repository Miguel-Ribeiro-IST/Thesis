%%%%%%%%%%%%%%%%%%%%%%%%%%%%%%%%%%%%%%%%%%%%%%%%%%%%%%%%%%%%%%%%%%%%%%
%     File: ExtendedAbstract_abstr.tex                               %
%     Tex Master: ExtendedAbstract.tex                               %
%                                                                    %
%     Author: Andre Calado Marta                                     %
%     Last modified : 2 Dez 2011                                     %
%%%%%%%%%%%%%%%%%%%%%%%%%%%%%%%%%%%%%%%%%%%%%%%%%%%%%%%%%%%%%%%%%%%%%%
% The abstract of should have less than 500 words.
% The keywords should be typed here (three to five keywords).
%%%%%%%%%%%%%%%%%%%%%%%%%%%%%%%%%%%%%%%%%%%%%%%%%%%%%%%%%%%%%%%%%%%%%%

%%
%% Abstract
%%
\begin{abstract}

Volatility is one of the most important subjects in all of quantitative finance, due not only to its impact on the prices of options but also to its elusiveness. In this thesis we study some of the models most used to forecast this variable, namely Dupire's local volatility as well as Heston and Static/Dynamic SABR stochastic volatility models.
We train these models with some options' implied volatility data, making them able to replicate real market behavior. We find that, when dealing with options with a single maturity, the Static SABR model is the one that best fits the data, while with multiple maturities, the Heston model outperforms Dynamic SABR. All these models vastly outperform the constant volatility model, assumed in Black-Scholes.
We then use these trained models to price European and Barrier options with the Monte Carlo numerical pricing method, which is able to accurately predict implied volatilities for near-the-money options, failing for deep in-the-money European call options.
\\
%%
%% Keywords (max 5)
%%
\noindent{{\bf Keywords:}} Volatility, Option pricing, Dupire, Heston, Static SABR, Dynamic SABR \\

\end{abstract}

