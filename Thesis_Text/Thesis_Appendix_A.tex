%%%%%%%%%%%%%%%%%%%%%%%%%%%%%%%%%%%%%%%%%%%%%%%%%%%%%%%%%%%%%%%%%%%%%%%%
%                                                                      %
%     File: Thesis_Appendix_A.tex                                      %
%     Tex Master: Thesis.tex                                           %
%                                                                      %
%     Author: Andre C. Marta                                           %
%     Last modified :  2 Jul 2015                                      %
%                                                                      %
%%%%%%%%%%%%%%%%%%%%%%%%%%%%%%%%%%%%%%%%%%%%%%%%%%%%%%%%%%%%%%%%%%%%%%%%

\chapter{Dupire's Formula Derivation}
\label{chapter:dupireformuladerivation}
Here is presented a brief demonstration of Dupire's formula, as shown in eq. \eqref{dupire}.

In his article, Dupire begins by assuming that the stock price $S$ follows a dynamic transition probability density function $p(S(t),t,S'(t'),t')$. In other words, integrating this function would result in the probability of the stock price reaching a price $S'$ at a time $t'$ having started at $S$ at time $t$.

The present value of a call option, $C(S,t,K,T)$, can be deduced as its expected future payoff, discounted backwards in time, which results in
\begin{equation}
\begin{split}\label{deriv0}
C(K,T)=e^{-r(T-t)}\mathbb{E}\left[\max\left(S'-K,0\right)\right]&=e^{-r(T-t)}\int_0^\infty\max\left(S'-K,0\right)p(S,t,S',T)dS'\\
&=e^{-r(T-t)}\int_K^\infty(S'-K)p(S,t,S',T)dS'.
\end{split}
\end{equation}
Deriving this result once with respect to the strike price $K$, we obtain
\begin{equation}
\pdv{C}{K}=-e^{-r(T-t)}\int_K^\infty p(S,t,S',T)dS'.
\end{equation}
Deriving again with respect to the same variable results in
\begin{equation}
\pdv{^2C}{K^2}=e^{-r(T-t)}p(S,t,S',T).
\end{equation}

Due to its stochastic nature, the transition probability density function follows the Fokker-Planck equation, given by
\begin{equation}\label{FokkerPlanck}
\pdv{p}{T}=\frac{1}{2}\sigma^2\pdv{^2(S^2p)}{S^2}-r\pdv{(Sp)}{S}.
\end{equation}
\noindent with $\sigma$ our, still unknown, function of $S$ and $t$, evaluated at $t=T$.

From eq. \ref{deriv0} we can easily derive
\begin{equation}
\pdv{C}{T}=-rC+e^{-r(T-t)}\int_K^\infty(S'-K)\pdv{p}{T}dS'.
\end{equation}
Using eq. \ref{FokkerPlanck}, we can transform this relation into
\begin{equation}
\pdv{C}{T}=-rC+e^{-r(T-t)}\int_K^\infty(S'-K)\left(\frac{1}{2}\sigma^2\pdv{^2(S'^2p)}{S'^2}-r\pdv{(S'p)}{S'}\right)dS'.
\end{equation}
Integrating twice by parts and collecting all terms, we get
\begin{equation}
\pdv{C}{T}=\frac{1}{2}\sigma^2K^2\pdv{^2C}{K^2}-rK\pdv{C}{K}.
\end{equation}
Rearranging all terms, we are left with the Dupire's formula
\begin{equation}
\sigma=\sqrt{\frac{\displaystyle\pdv{C}{T}+rK\pdv{C}{K}}{\displaystyle\frac{1}{2}K^2\pdv{^2C}{K^2}}}.
\end{equation}



