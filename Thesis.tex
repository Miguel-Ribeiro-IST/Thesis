\documentclass[a4paper,prd,nofootinbib,superscriptaddress,floatfix]{revtex4}
%\documentclass[prd,twocolumn,nofootinbib,showpacs]{revtex4-1}

%\usepackage{amsmath,amssymb}
\usepackage{cancel}
\usepackage{graphicx}
\usepackage{epsfig}
\usepackage{amsmath,amsfonts,amssymb}
\usepackage{color,soul}
\newcommand{\myexp}{e^}
\usepackage{float}
\usepackage{graphicx}
\graphicspath{ {Images/} }
\usepackage[utf8]{inputenc}
\usepackage[compatibility=false]{caption}
\usepackage{subcaption}
\usepackage{eurosym}
\usepackage{siunitx}


\begin{document}

\title{Sensitivity Analysis on Least-Squares American Options Pricing}

\author{Miguel Ângelo Maia Ribeiro (n.79013)}

\affiliation{Departamento de F\'{\i}sica, Instituto Superior T\'ecnico, Universidade de Lisboa, Lisboa, Portugal}
%\begin{document}



\maketitle
\section{Introduction}
The stock market has suffered a complete paradigm shift in recent decades. The surge of computer science and mathematical finance has greatly enhanced our abilities to predict stock price changes sometimes long before they occur.

With the colossal sums handled daily in the stock market, even a small improvement on the predictive abilities of a given algorithm can lead to enormous profits for investors. Thus, it should be clear that a great amount of resources should be invested in developing these algorithms to perfection. An investor that does not follow this strategy would surely lose major resources when faced against his better prepared counterparts.

In the stock market, contrary to common belief, not only stocks are exchanged. The derivatives market has become increasingly important in recent decades. In fact, this market alone is responsible for several trillion dollars worth of trades anually.

One of the most important types of derivatives are options, with the most used being call and put options. In short, an option gives me the right to buy (in the case of a call option) or sell (in case of a put option) a particular stock for a fixed price at some point in the future. If the price of the stock goes up (call) or down (put), I can simply buy the said stock in the stock market for the lower fixed price, and sell it immediately after (called exercising the option), earning the difference.

Options have several advantages. For one, they limit the exposure to risk. If the price of a given stock were to crash, stock holders would lose a serious amount of funds whereas the owner of a call option could simply choose to not buy the option. On the other hand, they can also lead to great profits. With the same example, with the crash of a stock price, the owner of a put option could buy this option for the lower price (after the crash) and sell it for the higher price, fixed at the time of acquisition.

These derivatives also have their own caveats. The main disadvantage is fact that options always deal with stock prices in the future, one can never precisely predict how much will be earned, if anything at all. Thus, when dealing with options, stock price prediction is of the highest importance.

Options are also separated into different categories. The two best known are european and american options. An european option can only be exercised at a fixed date in the future, whereas an american option can be exercised anytime up until a fixed date.
It is clear that american options are pose a greater challenge to deal with. One can never be absolutely sure when exercising the option is the optimal decision. Immediately after the exercise, the stock price could change so that it would have been better to wait a while longer.
American options are nonetheless very much used by investors, so it is absolutely critical to understand which factors influence these types of derivatives.

Due to their high importance, options have been studied in detail in the past.
Robert Merton and Myron Scholes earned the 1997 Nobel prize in Economics for developing a mathematical model to predict the price of european options, the famous Black-Scholes formula.
Models to price american options, derived from the results of Black and Scholes, have also been proposed, among which the work of Longstaff and Schwartz are paramount.


Nowadays, computers have become fundamental tools in the study of stock




\noindent Derivatives have become increasingly important in recent decades, with the sums currently handled in these markets amounting to several trillion dollars.\\
For this reason, it is of the utmost importance to be able to accurately predict the payoff of such contracts to correctly price them.\\
In this thesis we will study particularly the pricing of American options. These contracts are especially difficult to price due to the high uncertainty associated with optimal stopping.\\
Several methods have been suggested to achieve this goal, such as finite difference, but we shall use the procedure proposed by Longstaff et al., using Monte Carlo simulation and least-squares regression.\\
We will mainly focus on the sensitivity analysis of this pricing process, namely how the parameters’ volatility affects the final calculated price.\\
This analysis is particularly important due to the sometimes-large uncertainty associated with the parameters used. For this reason, we find that the least-squares approach is ideal under such conditions, since we expect it to be less sensitive to volatility than other methods.

\section{Objectives}

\section{State of the Art}

\section{Bibliography}

\section{Timetable}

\iffalse
\section{Thesis Supervisors}
\begin{itemize}
  \item Cláudia Nunes Philippart, \ \ cnunes@math.tecnico.ulisboa.pt
  \item Rui Manuel Agostinho Dilão, \ \ ruidilao@tecnico.ulisboa.pt
  \item Claude Yves Cochet, \ \ claude.cochet@bnpparibas.com
\end{itemize}
\fi

\section{Main Bibliography}
\begin{itemize}
\item Hull, J. (2012). \textit{Options, futures, and other derivatives}. Boston: Prentice Hall.

\item Longstaff, F. A., \& Schwartz, E. S. (2001, January 1). Valuing American Options by Simulation: A Simple Least-Squares Approach. \textit{The Review of Financial Studies, 14}(1), 113-147.

\item Choe, G. H. (2016). \textit{Stochastic Analysis for Finance with Simulations}. Universitext. Springer International Publishing. 
\end{itemize}
\end{document}

