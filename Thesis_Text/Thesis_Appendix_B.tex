%%%%%%%%%%%%%%%%%%%%%%%%%%%%%%%%%%%%%%%%%%%%%%%%%%%%%%%%%%%%%%%%%%%%%%%%
%                                                                      %
%     File: Thesis_Appendix_B.tex                                      %
%     Tex Master: Thesis.tex                                           %
%                                                                      %
%     Author: Andre C. Marta                                           %
%     Last modified :  2 Jul 2015                                      %
%                                                                      %
%%%%%%%%%%%%%%%%%%%%%%%%%%%%%%%%%%%%%%%%%%%%%%%%%%%%%%%%%%%%%%%%%%%%%%%%

\chapter{CMA-ES Algorithm Formulas}
\label{chapter:CMAESAlg}

Here we present the formulas required for the calculation of the mean vector and the covariance matrix, to be used on the multivariate normal distribution, at each iteration of the CMA-ES optimization algorithm.

\section{The Optimization Algorithm}
\subsection{Initialization}
We initialize the algorithm by setting the first mean vector, $\mathbf{m}^{(0)}$, to some initial guess and the covariance matrix to the unit matrix, $\mathbf{C}^{(0)}=\mathbf{I}$.

\subsection{Sampling}
We sample $\lambda$ points, $\mathbf{y}_i^{(1)},\ i=1,\ldots,\lambda$, from the multivariate normal distribution $N(\mathbf{x};\mathbf{0},\mathbf{C}^{(0)})$, generating the first candidate solutions
\begin{equation}
\mathbf{x}^{(1)}_i=\mathbf{m}^{(0)}+\sigma^{(0)}\mathbf{y}^{(1)}_i, \ \ \ \ \ i=1,\ldots,\lambda,
\end{equation} 
\noindent where $\sigma^{(0)}=1$.

\subsection{Classification}
The candidate solutions are ordered based on their cost function, such that we denote $\mathbf{x}^{(1)}_{i:\lambda}$ as the $i$-th best classified point from the set $\mathbf{x}^{(1)}_1,\ldots,\mathbf{x}^{(1)}_\lambda$. In other words, $f(\mathbf{x}_{1:\lambda}^{(1)})\leq f(\mathbf{x}_{2:\lambda}^{(1)})\leq\ldots\leq f(\mathbf{x}_{\lambda:\lambda}^{(1)})$ (for simplicity, we refer here to the cost function as $f(\cdot)$).

\subsection{Selection}
From the ordered set $\mathbf{x}^{(1)}_{i:\lambda}$ we choose the first $\mu$ data points (with the lowest cost) and discard the others.
We then define the weights $\omega_i$ as
\begin{equation}
\omega_i=\frac{\left(\log\left(\mu+1/2\right)-\log(i)\right)}{\sum_{i=1}^\mu\left(\log\left(\mu+1/2\right)-\log(i)\right)}, \ \ \ \ \ i=1,\ldots,\mu.
\end{equation}

As an alternative we can also use $\omega_i=1/\mu$.


\subsection{Adaptation}
We are finally able to calculate the new mean vector and covariance matrix using
\begin{equation}
\left\langle\mathbf{y}^{(k)}\right\rangle_w=\sum_{i=1}^\mu\omega_i\mathbf{y}^{(k)}_{i:\lambda},
\end{equation}
\begin{equation}
\mathbf{m}^{(k)}=\mathbf{m}^{(k-1)}+\sigma^{(k-1)}\left\langle\mathbf{y}^{(k)}\right\rangle_w=\sum_{i=1}^\mu\omega_i\mathbf{x}^{(k)}_{i:\lambda},
\end{equation}
\begin{equation}
\mathbf{p}^{(k)}_\sigma=(1-c_\sigma)\mathbf{p}^{(k-1)}_\sigma+\sqrt{c_\sigma(2-c_\sigma)\mu_{\mathrm{eff}}}\left(\mathbf{C}^{(k-1)}\right)^{-1/2}\left\langle\mathbf{y}^{(k)}\right\rangle_w,
\end{equation}
\begin{equation}
\sigma^{(k)}=\sigma^{(k-1)}\mathrm{exp}\left(\frac{c_\sigma}{d_\sigma}\left(\frac{\|\mathbf{p}^{(k)}_\sigma\|}{E^*}-1\right)\right),
\end{equation}
\begin{equation}
\mathbf{p}^{(k)}_c=(1-c_c)\mathbf{p}^{(k-1)}_c+h_\sigma^{(k)}\sqrt{c_c(2-c_c)\mu_{\mathrm{eff}}}\left\langle\mathbf{y}^{(k)}\right\rangle_w,
\end{equation}
\begin{equation}
\mathbf{C}^{(k)}=\left(1-c_1-c_\mu\right)\mathbf{C}^{(k-1)}+c_1\left(\mathbf{p}_c^{(k)}\left(\mathbf{p}_c^{(k)}\right)^T+\delta\left(h_\sigma^{(k)}\right)\mathbf{C}^{(k-1)}\right)+c_\mu\sum_{i=1}^\mu\omega_i\mathbf{y}^{(k)}_{i:\lambda}\left(\mathbf{y}^{(k)}_{i:\lambda}\right)^T,
\end{equation}
\noindent where we iterate $k$ until the termination criterion is met and using
\begin{equation}
\mu_{\mathrm{eff}}=\left(\sum_{i=1}^\mu\omega_i^2\right)^{-1},
\end{equation}
\begin{equation}
c_c=\frac{4+\mu_{\mathrm{eff}}/D}{D+4+2\mu_{\mathrm{eff}}/D},
\end{equation}
\begin{equation}
c_\sigma=\frac{\mu_{\mathrm{eff}}+2}{D+\mu_{\mathrm{eff}}+5},
\end{equation}
\begin{equation}
d_\sigma=1+2\max\left(0,\ \sqrt{\frac{\mu_{\mathrm{eff}}-1}{D+1}}-1\right)+c_\sigma,
\end{equation}
\begin{equation}
c_1=\frac{2}{(D+1.3)^2+\mu_{\mathrm{eff}}},
\end{equation}
\begin{equation}
c_\mu=\min\left(1-c_1,\ 2\frac{\mu_{\mathrm{eff}}-2+1/\mu_{\mathrm{eff}}}{(D+2)^2+\mu_{\mathrm{eff}}}\right),
\end{equation}
\begin{equation}
E^*=\frac{\sqrt{2}\Gamma\left(\frac{D+1}{2}\right)}{\Gamma\left(\frac{D}{2}\right)},
\end{equation}
\begin{equation}h_\sigma^{(k)}=
\begin{cases} 
      1, & \mathrm{if} \frac{\|\mathbf{p}^{(k)}_\sigma\|}{\sqrt{1-\left(1-c_\sigma\right)^{2(k+1)}}}<\left(1.4+\frac{2}{D+1}\right)E^*\\
      0, & \mathrm{otherwise}
   \end{cases},
\end{equation}
\begin{equation}
\delta\left(h_\sigma^{(k)}\right)=\left(1-h_\sigma^{(k)}\right)c_c\left(2-c_c\right),
\end{equation}
\begin{equation}
\left(\mathbf{C}^{(k)}\right)^{-1/2}=\mathbf{B}\left(\mathbf{D}^{(k)}\right)^{-1}\mathbf{B}^T,
\end{equation}
\noindent where $D$ corresponds to the number of parameters of the model (i.e. the dimensions of the search space).












