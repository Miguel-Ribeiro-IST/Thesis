\documentclass[a4paper,prd,nofootinbib,superscriptaddress,floatfix]{revtex4}
%\documentclass[prd,twocolumn,nofootinbib,showpacs]{revtex4-1}

%\usepackage{amsmath,amssymb}
\usepackage{cancel}
\usepackage{graphicx}
\usepackage{epsfig}
\usepackage{amsmath,amsfonts,amssymb}
\usepackage{color,soul}
\newcommand{\myexp}{e^}
\usepackage{float}
\usepackage{graphicx}
\graphicspath{ {Images/} }
\usepackage[utf8]{inputenc}
\usepackage[compatibility=false]{caption}
\usepackage{subcaption}
\usepackage{eurosym}
\usepackage{siunitx}


\begin{document}

\title{Sensitivity Analysis on Least-Squares American Options Pricing}

\author{Miguel Ângelo Maia Ribeiro (n.79013)}

\affiliation{Departamento de F\'{\i}sica, Instituto Superior T\'ecnico, Universidade de Lisboa, Lisboa, Portugal}
%\begin{document}



\maketitle
\section{Thesis Summary}
\noindent Derivatives have become increasingly important in recent decades, with the sums currently handled in these markets amounting to several trillion dollars.\\
For this reason, it is of the utmost importance to be able to accurately predict the payoff of such contracts to correctly price them.\\
In this thesis we will study particularly the pricing of American options. These contracts are especially difficult to price due to the high uncertainty associated with optimal stopping.\\
Several methods have been suggested to achieve this goal, such as finite difference, but we shall use the procedure proposed by Longstaff et al., using Monte Carlo simulation and least-squares regression.\\
We will mainly focus on the sensitivity analysis of this pricing process, namely how the parameters’ volatility affects the final calculated price.\\
This analysis is particularly important due to the sometimes-large uncertainty associated with the parameters used. For this reason, we find that the least-squares approach is ideal under such conditions, since we expect it to be less sensitive to volatility than other methods.


\section{Thesis Supervisors}
\begin{itemize}
  \item Cláudia Nunes Philippart, \ \ cnunes@math.tecnico.ulisboa.pt
  \item Rui Manuel Agostinho Dilão, \ \ ruidilao@tecnico.ulisboa.pt
  \item Claude Yves Cochet, \ \ claude.cochet@bnpparibas.com
\end{itemize}

\section{Main Bibliography}
\begin{itemize}
\item Hull, J. (2012). \textit{Options, futures, and other derivatives}. Boston: Prentice Hall.

\item Longstaff, F. A., \& Schwartz, E. S. (2001, January 1). Valuing American Options by Simulation: A Simple Least-Squares Approach. \textit{The Review of Financial Studies, 14}(1), 113-147.

\item Choe, G. H. (2016). \textit{Stochastic Analysis for Finance with Simulations}. Universitext. Springer International Publishing. 
\end{itemize}
\end{document}

