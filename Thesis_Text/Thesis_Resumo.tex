%%%%%%%%%%%%%%%%%%%%%%%%%%%%%%%%%%%%%%%%%%%%%%%%%%%%%%%%%%%%%%%%%%%%%%%%
%                                                                      %
%     File: Thesis_Resumo.tex                                          %
%     Tex Master: Thesis.tex                                           %
%                                                                      %
%     Author: Andre C. Marta                                           %
%     Last modified :  2 Jul 2015                                      %
%                                                                      %
%%%%%%%%%%%%%%%%%%%%%%%%%%%%%%%%%%%%%%%%%%%%%%%%%%%%%%%%%%%%%%%%%%%%%%%%

\section*{Resumo}

% Add entry in the table of contents as section
\addcontentsline{toc}{section}{Resumo}

Volatilidade é um dos tópicos mais importantes em matemática financeira, graças não só ao seu impacto nos preços de opções, mas também à sua indefinição. Nesta dissertação estudamos alguns dos modelos mais utilizados para prever esta variável, nomeadamente o modelo de volatilidade local de Dupire, bem como os modelos de volatilidade estocástica de Heston e Static/Dynamic SABR.
Treinamos estes modelos com dados de volatilidade implícita de algumas opções, tornando-os capazes de replicar o comportamento do mercado. Descobrimos que, quando apenas são usadas opções com uma única maturidade, o modelo Static SABR é o que melhor se ajusta aos dados, enquanto que quando se usam múltiplas maturidades, o modelo Heston supera o Dynamic SABR. Todos estes modelos têm um desempenho vastamente melhor do que o modelo de volatilidade constante.
De seguida, os modelos treinados são usados para avaliar opções europeias e de barreira, com o método de avaliação numérica de Monte Carlo. Este é capaz de prever com precisão as volatilidades implícitas para opções "near-the-money", falhando para opções de compra europeias "deep in-the-money".

\vfill

\textbf{\Large Palavras-chave:} Volatilidade, Avaliação de Opções, Dupire, Heston, Static SABR, Dynamic SABR




